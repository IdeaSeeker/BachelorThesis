\specialsection{Заключение}

Основным результатом работы является готовый к использованию инструмент статического анализа кода на языке Java. В процессе достижения этой цели были выполнены следующие задачи.

\begin{itemize}
    \item Проведён обзор существующих на рынке решений, в ходе которого были выявлены их ключевые недостатки "--- поверхностность проводимого анализа, а также большое количество ложноположительных срабатываний.
    \item Реализован модуль создания отчётов в формате SARIF на основе тестовых случаев, сгенерированных инструментом UnitTestBot.
    \item Разработан удобный пользовательский интерфейс для просмотра отчётов SARIF в среде разработки IntelliJ IDEA.
    \item В символьную виртуальную машину встроена техника taint-анализа, расширяющая её возможности. Реализованный алгоритм позволяет находить нарушения безопасности, а именно, случаи использования непроверенных данных в критических секциях программы.
    \item Разработанный статический анализатор был протестирован на нескольких известных проектах с открытым исходным кодом, а также на решениях задач с сайта Codeforces. Проведённые эксперименты показали возможность применения продукта на практике.
\end{itemize}
