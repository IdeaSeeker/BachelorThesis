\section{Плагин для IntelliJ IDEA}

Абсолютное большинство программистов используют различные интегрированные среды разработки. Ключевой особенностью современных IDE является возможность подключения в них плагинов "--- сторонних модулей, выполняющих определённые задачи. Таким образом, каждый пользователь может установить себе нужный набор плагинов и решать возникающие при программировании задачи прямо в IDE, что часто удобнее, чем пользоваться отдельными сервисами.

Многие статические анализаторы имеют собственные плагины под ту или иную среду разработки. В данной работе было принято решение реализовать плагин для IntelliJ IDEA, который бы позволял запускать статический анализ и просматривать его результаты, то есть отчёт SARIF. Разработка, однако, велась не с нуля, а на основе уже существующего плагина для автоматической генерации модульных тестов "--- UnitTestBot.

\subsection{Детали реализации}

Платформа IntelliJ предоставляет сторонним разработчикам широкие возможности по созданию новых модулей \cite{idea-plugins}. В частности, существует стандартный механизм \verb|Inspections|, который позволяет отображать список найденных проблем на панели \verb|Problems View|. Для того чтобы воспользоваться этой возможностью, были выполнены следующие шаги.

\begin{enumerate}
    \item Зарегистрирован новый \verb|InspectionTool| в конфигурации плагина.
    \item Переопределён метод \verb|GlobalSimpleInspectionTool.checkFile|, который добавляет обнаруженные ошибки для конкретного переданного файла \var{psiFile} в общий список проблем для \var{psiFile}.
    \item Добавлен автоматический запуск механизма \verb|Inspections| сразу после генерации тестов и создания отчёта SARIF, чтобы отобразить найденные дефекты в IDE.
\end{enumerate}

Ниже представлен сценарий использования модифицированного плагина.

\begin{enumerate}
    \item Пользователь запускает генерацию тестов в специальном режиме, который предполагает дальнейшее отображение результатов статического анализа.
    \item После окончания работы символьной машины и создания отчёта SARIF, на экране открывается вкладка \verb|Problems View|, где перечислены обнаруженные инструментом ошибки и уязвимости в виде списка сообщений (поле \var{message} в SARIF).
    \item При нажатии на любой из элементов этого списка, IDE показывает соответствующую ему строчку кода (поле \var{location}).
    \item В окошке справа появляется кнопка для перехода к нужному сгенерированному тесту (поле \var{relatedLocation}).
    \item Также, есть возможность посмотреть трассировку стека найденного исключения (поле \var{codeFlows}). Причём эта функциональность поддерживает навигацию по коду для каждого представленного кадра стека.
\end{enumerate}

Отдельно отметим, что перед разработкой самого интерфейса, было исследовано несколько альтернативных путей решения задачи по визуализации результатов. Например, для этого можно было написать отдельный плагин или сверстать собственную вкладку. В итоге было решено, что способ с использованием уже готовых \verb|Inspections| и \verb|Problems View| будет наиболее прост в реализации и, при этом, привычен для конечного пользователя, поэтому был выбран именно он.

Помимо описанной функциональности, плагин имеет дополнительные возможности для повышения удобства использования. Например, существует возможность запускать анализ не только на одном выбранном классе, но и на нескольких, а также на всём проекте сразу.

\subsection{Пример работы}

В приложении на Рис. \ref{problems-view} представлен снимок экрана среды IntelliJ IDEA с разработанным пользовательским интерфейсом визуализации результатов статического анализа. В данном случае, инструмент был запущен на методах \var{Main.example} и \var{Util.abs}, реализации которых уже появлялись в примерах ранее, а также на методе \var{Main.sumAbs}.

\begin{code}
int sumAbs(int a, int b) {
    return Util.abs(a) + Util.abs(b);
}
\end{code}

При нажатии на кнопку \verb|View generated test|, происходит переход к соответствующему тесту, как показано в приложении на Рис. \ref{problems-view-test}. 

При нажатии на кнопку \verb|Analyze stack trace|, открывается вкладка, представленная в приложении на Рис. \ref{problems-view-stacktrace}.
